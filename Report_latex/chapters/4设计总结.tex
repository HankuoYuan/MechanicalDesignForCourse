\section{设计总结}
压力计作为一种普遍应用于工业领域的工具,专门用于测定气体的压力,其在工业制造中的重要性不容小觑。在真实的生产场景下,对压力计的性能标准极为严苛,强调高精确度、高灵敏度以及高度的安全可靠性。此外,鉴于不同的工作需求和环境挑战,压力计还需满足一系列特定要求,比如能够承受高温、抵抗振动、防止湿气侵入及防尘等。
\newline

本项目的设计目标聚焦于一款适用于常规工业环境的气体压力测量设备。我们提出的解决方案采用的是弹簧管式压力表,这种选择得益于其诸多优势:高度的灵敏性、成本效率、简约的构造、平滑稳定的传动机制,以及便于加工制造和广泛适用性。通过全面的设计与计算验证,可以确认该设计方案已基本满足既定的设计标准,圆满实现了预定的设计目标。
\newline

此外,本设计亦非无瑕,它未具体限定使用环境及待测气体类型,忽视了这些因素可能对压力表性能产生的影响。更进一步,环境因素如温湿度变化,以及气体本身对压力表材料的潜在腐蚀作用,都可能是导致测量精度下降的不可忽视的因素。
\newline

在过去两周的课程项目期间,我成功地将机械部件知识与制图技能融为一体,这一整合不仅增强了我的综合素养,还提升了我独立作业的能力。同时,我借此机会初探了机械仪表的设计窍门,并在此过程中逐步建立了正确且全面的设计理念。
\newline

在整个工作推进中,我有幸得到了指导教师与同学们的大力协助与宝贵支持,对此,我衷心地表示感谢。